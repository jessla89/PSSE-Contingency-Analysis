\documentclass{report}
\usepackage{graphicx}
\usepackage{hyperref}
\usepackage[left=2cm,right=2cm,top=2cm,bottom=1.5cm]{geometry}
\usepackage{booktabs}
\usepackage{array}
\usepackage{caption}
\usepackage[list=true,listformat=simple]{subcaption}
\usepackage{multirow}
\usepackage{setspace}
\usepackage{titlesec}
\usepackage{pifont}
\usepackage{lscape}
\usepackage{url}
\usepackage{hyperref}
\usepackage{comment}
\usepackage{array}
\usepackage{float}
\newcolumntype{P}[1]{>{\centering\arraybackslash}p{#1}}

\usepackage{listings}
\usepackage{color}
\usepackage{pythonhighlight}

\setcounter{secnumdepth}{4}
\makeatletter
\setlength\@fptop{0pt}             % default: '0\p@ \@plus 1fil'
\setlength\@fpsep{2\baselineskip}  % default: '8\p@ \@plus 2fil'
\makeatother


% Title Page
\title{PSSE Contingency Analysis}
\author{Jessla Varaparambil Abdul Kadher}
\begin{document}
\maketitle


\tableofcontents
\listoffigures
\listoftables

\newpage

\chapter{Introduction}

\section{Study Description}

The Year 1 Topology 1 contingency analysis of the hypothetical SAVNW system is carried out to report the branches reporting loading greater than 100\% and buses reporting upper and lower voltage limit violations. 

\begin{figure}[h!]
   \centering 
   \includegraphics[scale=0.5]{y1t1}
   \caption{Single Line Diagram Year 1, Topology 1} 
   \label{fig:y1t1}
\end{figure}

For the hypothetical SAVNW system, in year 1, the solar farm in area 2 is operational. The solar farm has an installed capacity of 117 MW. For the Year 1 - Topology 1, there are 7 generators in 3 areas. Total installed capacity of the system is a combination of firm capacity provided by already existing 6 generating units of 4153.25 MW and the non-firm capacity 117 MW added to the system by the solar farm at Bus 103. Solar being an intermittent resource, the output is varying with no output expected from it at night. In this study the intermittent behaviour of solar farm in taken into account by considering 3 output cases. A 0 MW output case is considered when there is no output from solar farm along with an average expected output of 50 MW and a maximum expected output of 100 MW. For Year 1 topology in addition to the three generation scenarios, three forecasted load scenarios are also studied. 

Detailed analysis of Year 1 Topology 1 Base Case was carried out. Analysis of system totals by area, generator contributions to each scenario and the considered load scenarios can be found \href{https://htmlpreview.github.io/?https://github.com/jessla89/PSSE-Data-Extraction/blob/main/Area_totals_Topology_1.html}{here}. The overload and voltage violations can be found \href{https://htmlpreview.github.io/?https://github.com/jessla89/PSSE-Data-Extraction/blob/main/limit_checking_Topology_1.html}{here}. 

\section{Contingency Analysis}


AC contingency calculation was conducted on the hypothetical SAVNW system for 9 different scenarios and hence on 9 different case files corresponding to Topology 1. The same configuration files are used for the 9 scenarios and are:
\begin{itemize}
\item Subsystem file savnw.sub - Studied subsystems of the studied scenario/ case are defined via the Subsystem Definition data file (Figure ~\ref{fig:sub})
\item Monitor file savnw.mon - Monitored Element Data File identifies the branches that are to be monitored for flow violations and the buses that are to be monitored for voltage violations (Figure ~\ref{fig:mon})
\item  Contingency file savnw.con - Contingency cases that are to be tested are defined in the Contingency Definition data file (Figure ~\ref{fig:con})
\end{itemize}

\begin{figure}[H]
   \centering 
   \includegraphics[scale=0.6]{sub.png}
   \caption{The subsystem file corresponding to Year 1 Topology 1} 
   \label{fig:sub}
\end{figure}

\begin{figure}[H]
   \centering 
   \includegraphics[scale=0.6]{mon.png}
   \caption{The monitored file corresponding to Year 1 Topology 1} 
   \label{fig:mon}
\end{figure}

\begin{figure}[H]
   \centering 
   \includegraphics[scale=0.6]{con.png}
   \caption{The contingency file corresponding to Year 1 Topology 1} 
   \label{fig:con}
\end{figure}


For each of the 9 studied scenarios, the API DFAX\_2 is used to construct 9 different distribution factor data files corresponding to each .sav file, and the above defined .sub, .mon, .con configuration files.
For each of the 9 scenarios, by running the AC contingency calculation function ACCC\_WITH\_DSP\_3, the contingency solution output .acc files are obtained.

Python code to conduct AC contingency calculation is:


\begin{python}
import psspy
list_gens = [0,50,100]
list_lsc = ['lls','rls','hls']
for gen in list_gens:
    for lsc in list_lsc:
        file_in = 'sav\savnw_sol_' + str(gen) +'_'+ lsc +'.sav'
        file_dist = 'savnw_sol_' + str(gen) +'_'+ lsc + '.dfx'
        file_out = 'savnw_sol_' + str(gen) +'_'+ lsc + '.acc'
        file_sav = r"{}".format(file_in)
        file_dfx = r"{}".format(file_dist)
        file_acc = r"{}".format(file_out)
        psspy.case(file_in)
        psspy.fdns([0,1,0,0,0,0,0,0])
        psspy.dfax_2([1,1,0],r"""savnw.sub""",r"""savnw.mon""",r"""savnw.con""",file_dfx)
        psspy.accc_with_dsp_3(0.1,[0,1,0,0,0,0,0,0,0,0,0],"",file_dfx,file_acc,"","","")
\end{python}

For each of the 9 scenarios, using the contingency solution output files, the results are exported as excel files for further analysis. The results exported are ACCC Analysis Summary, Monitored Branch Flows (MVA), Monitored Bus Voltages. 

Python code to export AC contingency solution output file as excel is: 

\begin{python}
import psspy
import pssexcel
list_gens = [0,50,100]
list_lsc = ['lls','rls','hls']
for gen in list_gens:
    for lsc in list_lsc:
        file_in = 'acc\savnw_sol_' + str(gen) +'_'+ lsc + '.acc'
        file_out = 'savnw_sol_' + str(gen) +'_'+ lsc + '.xlsx'
        file_acc = r"{}".format(file_in)
        file_xlsx = r"{}".format(file_out)
        pssexcel.accc(file_acc, ['s','v','g','l','b','i','n','w'], colabel='',stype='contingency', busmsm=0.5, sysmsm=5.0,
                       rating='a', namesplit=False,xlsfile=file_out, sheet='', overwritesheet=True, show=False, ratecon='b',
                       baseflowvio=True,basevoltvio=True, flowlimit=100.0, flowchange=0.0, voltchange=0.0,swdrating='a',
                       swdratecon='b',baseswdflowvio=False,basenodevoltvio=False,overloadreport=False)
\end{python}

Contingency analysis was carried out in PSSE for each of the studied scenario for Year 1 Topology 1. It was seen that the power flow solution did not converge for some of the tested contingencies for the studied scenario. For each of the studied scenario, the contingencies for which power flow solution did not converge are:

\begin{itemize}
\item Solar = 0 MW, LLS
\begin{itemize}
\item BUS 154, BUS 201, UNIT 206(1), BUS 152, SING OPN LIN   15 205-206(1)
\end{itemize}
\item Solar = 0 MW, RLS
\begin{itemize}
\item BUS 154, BUS 201, UNIT 206(1), BUS 152, BUS 151, SING OPN LIN   15 205-206(1)
\end{itemize}
\item Solar = 0 MW, HLS
\begin{itemize}
\item BUS 202, BUS 154, BUS 201, UNIT 206(1), BUS 152, BUS 151, SING OPN LIN   15 205-206(1)
\end{itemize}
\item Solar = 50 MW, LLS
\begin{itemize}
\item BUS 201, BUS 154, BUS 152
\end{itemize}
\item Solar = 50 MW, RLS
\begin{itemize}
\item BUS 154, BUS 201, UNIT 206(1), BUS 152, SING OPN LIN   15 205-206(1)
\end{itemize}
\item Solar = 50 MW, HLS
\begin{itemize}
\item BUS 154, BUS 201, UNIT 206(1), BUS 152, BUS 151, SING OPN LIN   15 205-206(1)
\end{itemize}
\item Solar = 100 MW, LLS
\begin{itemize}
\item BUS 201, BUS 154, BUS 152
\end{itemize}
\item Solar = 100 MW, RLS
\begin{itemize}
\item BUS 154, BUS 201, UNIT 206(1), BUS 152, SING OPN LIN   15 205-206(1)
\end{itemize}
\item Solar = 100 MW, HLS
\begin{itemize}
\item BUS 154, BUS 201, UNIT 206(1), BUS 152, BUS 151, SING OPN LIN   15 205-206(1)
\end{itemize}
\end{itemize}

For the converged contingencies for each of the studied scenario, the results of the contingency analysis were analysed to check for branch overload ($ > 100\%$) and out of range bus voltage violations (lower emergency limit $ < 0.9 PU $ and upper emergency limit $ > 1.1 PU$). It was seen that for some of the contingencies there were no branch overload or bus voltage violations reported. Rest of the contingencies violating branch overload and bus voltage emergency ranges are reported in the subsequent chapters. Chapter ~\ref{overload} of this document gives the observed branch flow violations for each of the studied scenario. Chapter ~\ref{lower} gives the lower voltage violations reported for each of the studied scenarios. Chapter ~\ref{upper} gives the upper voltage violations reported for each of the studied scenarios. To conclude, Chapter ~\ref{conclusion} summarises the result of the contingency analysis carried out on Year 1, Topology 1 of the hypothetical SAVNW system. 


\chapter{Branch Overload Violation}
\label{overload}
\section{Introduction}
In this chapter, for each of the studied scenario, the branches that are loaded more than 100\% of their rating is tabulated. Branches that are loaded more than 130\% of the rating are said to be severely/ critically loaded and are noted down for reporting as operating these branches for prolonged duration is not recommended for a safe and reliable power system.

\section{Solar = 0 MW, LLS}

For the studied scenario Solar = 0 MW, LLS, loading greater than 130\% were reported for the branch 3001-3003(1) for the unit fault contingency UNIT 101(1), for the branches 3001-3003(1), 3003-3005(1), 3003-3005(2) for the bus fault contingency BUS 151, and for the branch 3001-3003(1) for single line open contingencies SING OPN LIN   1 101-151(1).

\begin{table}[H]
\centering
\scalebox{0.9}{
\begin{tabular}{llrrrr}
\toprule
Branch & Contingency & MVA Flow & AMP Flow & Rate & Loading \\
\midrule
  3001     MINE        230.00   3003     S. MINE     230.00 1 & UNIT 101(1) & 513.17 & 496.15 & 300.00 & 165.38 \\
\bottomrule
\end{tabular}}
\caption{Unit faults reporting branch flow greater than 100\% for scenario Solar = 0 MW, LLS}
\label{unitSolar0MWLLS}
\end{table}

\begin{table}[H]
\centering
\scalebox{0.9}{
\begin{tabular}{llrrrr}
\toprule
Branch & Contingency & MVA Flow & AMP Flow & Rate & Loading \\
\midrule
  3001     MINE        230.00   3003     S. MINE     230.00 1 & BUS 151 & 1026.47 & 1030.17 & 300.00 & 343.39 \\
  3001     MINE        230.00   3011     MINE\_G      13.800 1 & BUS 151 & -1606.05 & 1606.05 & 1560.00 & 102.95 \\
  3003     S. MINE     230.00   3005     WEST        230.00 1 & BUS 151 & 505.93 & 515.08 & 350.00 & 147.17 \\
  3003     S. MINE     230.00   3005     WEST        230.00 2 & BUS 151 & 505.93 & 515.08 & 350.00 & 147.17 \\
  3005     WEST        230.00   3007     RURAL       230.00 1 & BUS 151 & -346.44 & 369.36 & 350.00 & 105.53 \\
   153     MID230      230.00    154     DOWNTN      230.00 1 & BUS 203 & -345.16 & 353.67 & 350.00 & 101.05 \\
   154     DOWNTN      230.00    203     EAST230     230.00 1 & BUS 205 & 284.68 & 318.15 & 250.00 & 127.26 \\
\bottomrule
\end{tabular}}
\caption{Bus faults reporting branch flow greater than 100\% for scenario Solar = 0 MW, LLS}
\label{busSolar0MWLLS}
\end{table}

\begin{table}[H]
\centering
\scalebox{0.9}{
\begin{tabular}{llrrrr}
\toprule
Branch & Contingency & MVA Flow & AMP Flow & Rate & Loading \\
\midrule
  3001     MINE        230.00   3003     S. MINE     230.00 1 & SING OPN LIN   1 101-151(1) & 513.17 & 496.15 & 300.00 & 165.38 \\
   153     MID230      230.00    154     DOWNTN      230.00 1 & SING OPN LIN   12 202-203(1) & -346.58 & 355.41 & 350.00 & 101.55 \\
\bottomrule
\end{tabular}}
\caption{Single line open contingencies reporting branch flow greater than 100\% for scenario Solar = 0 MW, LLS}
\label{singleSolar0MWLLS}
\end{table}

\section{Solar = 0 MW, RLS}
For the studied scenario Solar = 0 MW, RLS, loading greater than 130\% were reported for the branch 3001-3003(1) for the unit fault contingency UNIT 101(1) and single line open contingency SING OPN LIN   1 101-151(1), and for the branch 154-203(1) for the bus fault contingency BUS 205. 

\begin{table}[H]
\centering
\scalebox{0.9}{
\begin{tabular}{llrrrr}
\toprule
Branch & Contingency & MVA Flow & AMP Flow & Rate & Loading \\
\midrule
  3001     MINE        230.00   3003     S. MINE     230.00 1 & UNIT 101(1) & 568.29 & 550.89 & 300.00 & 183.63 \\
   205     SUB230      230.00    206     URBGEN      18.000 1 & UNIT 211(1) & -1253.09 & 1253.09 & 1250.00 & 100.25 \\
\bottomrule
\end{tabular}}
\caption{Unit faults reporting branch flow greater than 100\% for scenario Solar = 0 MW, RLS}
\label{unitSolar0MWRLS}
\end{table}


\begin{table}[H]
\centering
\scalebox{0.9}{
\begin{tabular}{llrrrr}
\toprule
Branch & Contingency & MVA Flow & AMP Flow & Rate & Loading \\
\midrule
   153     MID230      230.00    154     DOWNTN      230.00 1 & BUS 203 & -354.71 & 364.24 & 350.00 & 104.07 \\
   153     MID230      230.00    154     DOWNTN      230.00 1 & BUS 205 & -321.62 & 375.49 & 350.00 & 107.28 \\
   154     DOWNTN      230.00    203     EAST230     230.00 1 & BUS 205 & 297.88 & 347.78 & 250.00 & 139.11 \\
\bottomrule
\end{tabular}}
\caption{Bus faults reporting branch flow greater than 100\% for scenario Solar = 0 MW, RLS}
\label{busSolar0MWRLS}
\end{table}


\begin{table}[H]
\centering
\scalebox{0.9}{
\begin{tabular}{llrrrr}
\toprule
Branch & Contingency & MVA Flow & AMP Flow & Rate & Loading \\
\midrule
  3001     MINE        230.00   3003     S. MINE     230.00 1 & SING OPN LIN   1 101-151(1) & 568.29 & 550.89 & 300.00 & 183.63 \\
   154     DOWNTN      230.00    205     SUB230      230.00 1 & SING OPN LIN   6 152-153(1) & 672.26 & 694.04 & 660.00 & 105.16 \\
   153     MID230      230.00    154     DOWNTN      230.00 1 & SING OPN LIN   10 201-205(\&1) & -324.03 & 351.40 & 350.00 & 100.40 \\
   205     SUB230      230.00    206     URBGEN      18.000 1 & SING OPN LIN   11 201-211(1) & -1253.06 & 1253.06 & 1250.00 & 100.25 \\
   153     MID230      230.00    154     DOWNTN      230.00 1 & SING OPN LIN   12 202-203(1) & -356.57 & 366.46 & 350.00 & 104.70 \\
\bottomrule
\end{tabular}}
\caption{Single line open contingencies reporting branch flow greater than 100\% for scenario Solar = 0 MW, RLS}
\label{singleSolar0MWRLS}
\end{table}


\section{Solar = 0 MW, HLS}

For the studied scenario Solar = 0 MW, HLS, loading greater than 130\% were reported for the branch 3001-3003(1) for the unit fault contingencies UNIT 101(1) and single line open contingency SING OPN LIN   1 101-151(1), and for the branch 154-203(1) for the bus fault contingency BUS 205.

\begin{table}[H]
\centering
\scalebox{0.9}{
\begin{tabular}{llrrrr}
\toprule
Branch & Contingency & MVA Flow & AMP Flow & Rate & Loading \\
\midrule
  3001     MINE        230.00   3003     S. MINE     230.00 1 & UNIT 101(1) & 612.37 & 595.00 & 300.00 & 198.33 \\
   101     NUC-A       21.600    151     NUCPANT     500.00 1 & UNIT 102(1) & 1377.00 & 1377.00 & 1350.00 & 102.00 \\
   205     SUB230      230.00    206     URBGEN      18.000 1 & UNIT 211(1) & -1327.88 & 1327.88 & 1250.00 & 106.23 \\
\bottomrule
\end{tabular}}
\caption{Unit faults reporting branch flow greater than 100\% for scenario Solar = 0 MW, HLS}
\label{unitSolar0MWHLS}
\end{table}


\begin{table}[H]
\centering
\scalebox{0.9}{
\begin{tabular}{llrrrr}
\toprule
Branch & Contingency & MVA Flow & AMP Flow & Rate & Loading \\
\midrule
   154     DOWNTN      230.00    205     SUB230      230.00 1 & BUS 153 & 649.27 & 670.42 & 660.00 & 101.58 \\
   153     MID230      230.00    154     DOWNTN      230.00 1 & BUS 203 & -362.58 & 372.97 & 350.00 & 106.56 \\
   153     MID230      230.00    154     DOWNTN      230.00 1 & BUS 205 & -335.66 & 409.71 & 350.00 & 117.06 \\
   154     DOWNTN      230.00    203     EAST230     230.00 1 & BUS 205 & 308.70 & 376.80 & 250.00 & 150.72 \\
\bottomrule
\end{tabular}}
\caption{Bus faults reporting branch flow greater than 100\% for scenario Solar = 0 MW, HLS}
\label{busSolar0MWHLS}
\end{table}


\begin{table}[H]
\centering
\scalebox{0.9}{
\begin{tabular}{llrrrr}
\toprule
Branch & Contingency & MVA Flow & AMP Flow & Rate & Loading \\
\midrule
  3001     MINE        230.00   3003     S. MINE     230.00 1 & SING OPN LIN   1 101-151(1) & 612.37 & 595.00 & 300.00 & 198.33 \\
   101     NUC-A       21.600    151     NUCPANT     500.00 1 & SING OPN LIN   2 102-151(1) & 1377.00 & 1377.00 & 1350.00 & 102.00 \\
   154     DOWNTN      230.00    205     SUB230      230.00 1 & SING OPN LIN   6 152-153(1) & 717.09 & 750.02 & 660.00 & 113.64 \\
   153     MID230      230.00    154     DOWNTN      230.00 1 & SING OPN LIN   10 201-205(\&1) & -338.42 & 399.53 & 350.00 & 114.15 \\
   205     SUB230      230.00    206     URBGEN      18.000 1 & SING OPN LIN   11 201-211(1) & -1327.86 & 1327.86 & 1250.00 & 106.23 \\
   153     MID230      230.00    154     DOWNTN      230.00 1 & SING OPN LIN   12 202-203(1) & -372.18 & 406.86 & 350.00 & 116.25 \\
\bottomrule
\end{tabular}}
\caption{Single line open contingencies reporting branch flow greater than 100\% for scenario Solar = 0 MW, HLS}
\label{singleSolar0MWHLS}
\end{table}


\section{Solar = 50 MW, LLS}

For the studied scenario Solar = 50 MW, LLS, loading greater than 130\% were reported for the branch 3001-3003(1) for unit fault contingencies UNIT 101(1), UNIT 206(1), for single line open contingencies SING OPN LIN   1 101-151(1), SING OPN LIN   15 205-206(1), and for the branches 3001-3003(1), 3003-3005(1), 3003-3005(2) for the bus fault contingency BUS 151. 

\begin{table}[H]
\centering
\scalebox{0.9}{
\begin{tabular}{llrrrr}
\toprule
Branch & Contingency & MVA Flow & AMP Flow & Rate & Loading \\
\midrule
  3001     MINE        230.00   3003     S. MINE     230.00 1 & UNIT 101(1) & 479.02 & 462.62 & 300.00 & 154.21 \\
   153     MID230      230.00    154     DOWNTN      230.00 1 & UNIT 206(1) & -319.68 & 370.72 & 350.00 & 105.92 \\
  3001     MINE        230.00   3003     S. MINE     230.00 1 & UNIT 206(1) & 647.55 & 641.38 & 300.00 & 213.79 \\
  3005     WEST        230.00   3007     RURAL       230.00 1 & UNIT 206(1) & -325.74 & 359.78 & 350.00 & 102.79 \\
\bottomrule
\end{tabular}}
\caption{Unit faults reporting branch flow greater than 100\% for scenario Solar = 50 MW, LLS}
\label{unitSolar50MWLLS}
\end{table}


\begin{table}[H]
\centering
\scalebox{0.9}{
\begin{tabular}{llrrrr}
\toprule
Branch & Contingency & MVA Flow & AMP Flow & Rate & Loading \\
\midrule
  3001     MINE        230.00   3003     S. MINE     230.00 1 & BUS 151 & 985.23 & 984.52 & 300.00 & 328.17 \\
  3003     S. MINE     230.00   3005     WEST        230.00 1 & BUS 151 & 486.35 & 492.26 & 350.00 & 140.65 \\
  3003     S. MINE     230.00   3005     WEST        230.00 2 & BUS 151 & 486.35 & 492.26 & 350.00 & 140.65 \\
  3005     WEST        230.00   3007     RURAL       230.00 1 & BUS 151 & -336.71 & 356.53 & 350.00 & 101.86 \\
   154     DOWNTN      230.00    203     EAST230     230.00 1 & BUS 205 & 271.60 & 300.48 & 250.00 & 120.19 \\
\bottomrule
\end{tabular}}
\caption{Bus faults reporting branch flow greater than 100\% for scenario Solar = 50 MW, LLS}
\label{busSolar50MWLLS}
\end{table}


\begin{table}[H]
\centering
\scalebox{0.9}{
\begin{tabular}{llrrrr}
\toprule
Branch & Contingency & MVA Flow & AMP Flow & Rate & Loading \\
\midrule
  3001     MINE        230.00   3003     S. MINE     230.00 1 & SING OPN LIN   1 101-151(1) & 479.02 & 462.62 & 300.00 & 154.21 \\
   153     MID230      230.00    154     DOWNTN      230.00 1 & SING OPN LIN   15 205-206(1) & -319.68 & 370.72 & 350.00 & 105.92 \\
  3001     MINE        230.00   3003     S. MINE     230.00 1 & SING OPN LIN   15 205-206(1) & 647.55 & 641.38 & 300.00 & 213.79 \\
  3005     WEST        230.00   3007     RURAL       230.00 1 & SING OPN LIN   15 205-206(1) & -325.74 & 359.78 & 350.00 & 102.79 \\
\bottomrule
\end{tabular}}
\caption{Single line open contingencies reporting branch flow greater than 100\% for scenario Solar = 50 MW, LLS}
\label{singleSolar50MWLLS}
\end{table}


\section{Solar = 50 MW, RLS}

For the studied scenario Solar = 50 MW, RLS, loading greater than 130\% were reported for the branch 3001-3003(1) for the unit fault contingency UNIT 101(1), for the branches 3001-3003(1), 3003-3005(1), 3003-3005(2) for the bus fault contingency BUS 151, for the branch 154-203(1) for the bus fault contingency BUS 205, and for the branch 3001-3003(1) for single line open contingency SING OPN LIN   1 101-151(1).

\begin{table}[H]
\centering
\scalebox{0.9}{
\begin{tabular}{llrrrr}
\toprule
Branch & Contingency & MVA Flow & AMP Flow & Rate & Loading \\
\midrule
  3001     MINE        230.00   3003     S. MINE     230.00 1 & UNIT 101(1) & 533.75 & 516.73 & 300.00 & 172.24 \\
\bottomrule
\end{tabular}}
\caption{Unit faults reporting branch flow greater than 100\% for scenario Solar = 50 MW, RLS}
\label{unitSolar50MWRLS}
\end{table}

\begin{table}[H]
\centering
\scalebox{0.9}{
\begin{tabular}{llrrrr}
\toprule
Branch & Contingency & MVA Flow & AMP Flow & Rate & Loading \\
\midrule
   153     MID230      230.00   3006     UPTOWN      230.00 1 & BUS 151 & -329.34 & 377.95 & 350.00 & 107.99 \\
  3001     MINE        230.00   3003     S. MINE     230.00 1 & BUS 151 & 1117.38 & 1159.84 & 300.00 & 386.61 \\
  3001     MINE        230.00   3011     MINE\_G      13.800 1 & BUS 151 & -1805.06 & 1805.06 & 1560.00 & 115.71 \\
  3003     S. MINE     230.00   3005     WEST        230.00 1 & BUS 151 & -499.62 & 580.89 & 350.00 & 165.97 \\
  3003     S. MINE     230.00   3005     WEST        230.00 2 & BUS 151 & -499.62 & 580.89 & 350.00 & 165.97 \\
  3005     WEST        230.00   3006     UPTOWN      230.00 1 & BUS 151 & 326.17 & 379.22 & 350.00 & 108.35 \\
  3005     WEST        230.00   3007     RURAL       230.00 1 & BUS 151 & -354.86 & 424.54 & 350.00 & 121.30 \\
   153     MID230      230.00    154     DOWNTN      230.00 1 & BUS 203 & -344.74 & 353.82 & 350.00 & 101.09 \\
   153     MID230      230.00    154     DOWNTN      230.00 1 & BUS 205 & -309.30 & 356.38 & 350.00 & 101.82 \\
   154     DOWNTN      230.00    203     EAST230     230.00 1 & BUS 205 & 285.04 & 328.43 & 250.00 & 131.37 \\
\bottomrule
\end{tabular}}
\caption{Bus faults reporting branch flow greater than 100\% for scenario Solar = 50 MW, RLS}
\label{busSolar50MWRLS}
\end{table}

\begin{table}[H]
\centering
\scalebox{0.9}{
\begin{tabular}{llrrrr}
\toprule
Branch & Contingency & MVA Flow & AMP Flow & Rate & Loading \\
\midrule
  3001     MINE        230.00   3003     S. MINE     230.00 1 & SING OPN LIN   1 101-151(1) & 533.75 & 516.73 & 300.00 & 172.24 \\
   154     DOWNTN      230.00    205     SUB230      230.00 1 & SING OPN LIN   6 152-153(1) & 642.92 & 663.52 & 660.00 & 100.53 \\
   153     MID230      230.00    154     DOWNTN      230.00 1 & SING OPN LIN   12 202-203(1) & -346.37 & 355.80 & 350.00 & 101.66 \\
\bottomrule
\end{tabular}}
\caption{Single line open contingencies reporting branch flow greater than 100\% for scenario Solar = 50 MW, RLS}
\label{singleSolar50MWRLS}
\end{table}

\section{Solar = 50 MW, HLS}
\begin{table}[H]
\centering
\scalebox{0.9}{
\begin{tabular}{llrrrr}
\toprule
Branch & Contingency & MVA Flow & AMP Flow & Rate & Loading \\
\midrule
  3001     MINE        230.00   3003     S. MINE     230.00 1 & UNIT 101(1) & 577.57 & 560.36 & 300.00 & 186.79 \\
   205     SUB230      230.00    206     URBGEN      18.000 1 & UNIT 211(1) & -1323.48 & 1323.48 & 1250.00 & 105.88 \\
\bottomrule
\end{tabular}}
\caption{Unit faults reporting branch flow greater than 100\% for scenario Solar = 50 MW, HLS}
\label{unitSolar50MWHLS}
\end{table}

For the studied scenario Solar = 50 MW, HLS, loading greater than 130\% were reported for the branch 3001-3003(1) for the unit fault contingency UNIT 101(1), for the branch 154-203(1) for the bus fault contingency BUS 205, and for the branch 3001-3003(1) for single line open contingency SING OPN LIN   1 101-151(1).

\begin{table}[H]
\begin{table}[H]
\centering
\scalebox{0.9}{
\begin{tabular}{llrrrr}
\toprule
Branch & Contingency & MVA Flow & AMP Flow & Rate & Loading \\
\midrule
   153     MID230      230.00    154     DOWNTN      230.00 1 & BUS 202 & -320.34 & 376.82 & 350.00 & 107.66 \\
   153     MID230      230.00    154     DOWNTN      230.00 1 & BUS 203 & -352.26 & 362.17 & 350.00 & 103.48 \\
   153     MID230      230.00    154     DOWNTN      230.00 1 & BUS 205 & -323.57 & 388.14 & 350.00 & 110.90 \\
   154     DOWNTN      230.00    203     EAST230     230.00 1 & BUS 205 & 296.06 & 355.14 & 250.00 & 142.06 \\
\bottomrule
\end{tabular}}
\caption{Bus faults reporting branch flow greater than 100\% for scenario Solar = 50 MW, HLS}
\label{busSolar50MWHLS}
\end{table}


\centering
\scalebox{0.9}{
\begin{tabular}{llrrrr}
\toprule
Branch & Contingency & MVA Flow & AMP Flow & Rate & Loading \\
\midrule
  3001     MINE        230.00   3003     S. MINE     230.00 1 & SING OPN LIN   1 101-151(1) & 577.57 & 560.36 & 300.00 & 186.79 \\
   154     DOWNTN      230.00    205     SUB230      230.00 1 & SING OPN LIN   6 152-153(1) & 690.45 & 713.82 & 660.00 & 108.15 \\
   153     MID230      230.00    154     DOWNTN      230.00 1 & SING OPN LIN   10 201-205(\&1) & -326.81 & 376.29 & 350.00 & 107.51 \\
   205     SUB230      230.00    206     URBGEN      18.000 1 & SING OPN LIN   11 201-211(1) & -1323.45 & 1323.45 & 1250.00 & 105.88 \\
   153     MID230      230.00    154     DOWNTN      230.00 1 & SING OPN LIN   12 202-203(1) & -356.29 & 372.29 & 350.00 & 106.37 \\
\bottomrule
\end{tabular}}
\caption{Single line open contingencies reporting branch flow greater than 100\% for scenario Solar = 50 MW, HLS}
\label{singleSolar50MWHLS}
\end{table}


\section{Solar = 100 MW, LLS}


For the studied scenario Solar = 100 MW, LLS, loading greater than 130\% were reported for the branch 3001-3003(1) for the unit fault contingencies UNIT 101(1), UNIT 206(1), for the branches 3001-3003(1), 3003-3005(1), 3003-3005(2) for the bus fault contingency BUS 151, and for single line open contingencies SING OPN LIN   1 101-151(1) and SING OPN LIN   15 205-206(1).

\begin{table}[H]
\centering
\scalebox{0.9}{
\begin{tabular}{llrrrr}
\toprule
Branch & Contingency & MVA Flow & AMP Flow & Rate & Loading \\
\midrule
  3001     MINE        230.00   3003     S. MINE     230.00 1 & UNIT 101(1) & 445.39 & 429.71 & 300.00 & 143.24 \\
   153     MID230      230.00    154     DOWNTN      230.00 1 & UNIT 206(1) & -312.25 & 360.29 & 350.00 & 102.94 \\
  3001     MINE        230.00   3003     S. MINE     230.00 1 & UNIT 206(1) & 642.96 & 636.21 & 300.00 & 212.07 \\
  3005     WEST        230.00   3007     RURAL       230.00 1 & UNIT 206(1) & -322.06 & 354.27 & 350.00 & 101.22 \\
\bottomrule
\end{tabular}}
\caption{Unit faults reporting branch flow greater than 100\% for scenario Solar = 100 MW, LLS}
\label{unitSolar100MWLLS}
\end{table}

\begin{table}[H]
\centering
\scalebox{0.9}{
\begin{tabular}{llrrrr}
\toprule
Branch & Contingency & MVA Flow & AMP Flow & Rate & Loading \\
\midrule
  3001     MINE        230.00   3003     S. MINE     230.00 1 & BUS 151 & 937.87 & 932.95 & 300.00 & 310.98 \\
  3003     S. MINE     230.00   3005     WEST        230.00 1 & BUS 151 & 463.70 & 466.47 & 350.00 & 133.28 \\
  3003     S. MINE     230.00   3005     WEST        230.00 2 & BUS 151 & 463.70 & 466.47 & 350.00 & 133.28 \\
   154     DOWNTN      230.00    203     EAST230     230.00 1 & BUS 205 & 259.17 & 284.87 & 250.00 & 113.95 \\
\bottomrule
\end{tabular}}
\caption{Bus faults reporting branch flow greater than 100\% for scenario Solar = 100 MW, LLS}
\label{busSolar100MWLLS}
\end{table}


\begin{table}[H]
\centering
\scalebox{0.9}{
\begin{tabular}{llrrrr}
\toprule
Branch & Contingency & MVA Flow & AMP Flow & Rate & Loading \\
\midrule
  3001     MINE        230.00   3003     S. MINE     230.00 1 & SING OPN LIN   1 101-151(1) & 445.39 & 429.71 & 300.00 & 143.24 \\
   153     MID230      230.00    154     DOWNTN      230.00 1 & SING OPN LIN   15 205-206(1) & -312.25 & 360.29 & 350.00 & 102.94 \\
  3001     MINE        230.00   3003     S. MINE     230.00 1 & SING OPN LIN   15 205-206(1) & 642.96 & 636.21 & 300.00 & 212.07 \\
  3005     WEST        230.00   3007     RURAL       230.00 1 & SING OPN LIN   15 205-206(1) & -322.06 & 354.27 & 350.00 & 101.22 \\
\bottomrule
\end{tabular}}
\caption{Single line open contingencies reporting branch flow greater than 100\% for scenario Solar = 100 MW, LLS}
\label{singleSolar100MWLLS}
\end{table}


\section{Solar = 100 MW, RLS}

For the studied scenario Solar = 100 MW, RLS, loading greater than 130\% were reported for the branches 3001-3003(1) for the unit fault contingency UNIT 101(1), for the branches 3001-3003(1), 3003-3005(1), 3003-3005(2) for the bus fault contingency BUS 151, and for the branch 3001-3003(1) for single line open contingency SING OPN LIN   1 101-151(1).

\begin{table}[H]
\centering
\scalebox{0.9}{
\begin{tabular}{llrrrr}
\toprule
Branch & Contingency & MVA Flow & AMP Flow & Rate & Loading \\
\midrule
  3001     MINE        230.00   3003     S. MINE     230.00 1 & UNIT 101(1) & 499.44 & 482.94 & 300.00 & 160.98 \\
\bottomrule
\end{tabular}}
\caption{Unit faults reporting branch flow greater than 100\% for scenario Solar = 100 MW, RLS}
\label{unitSolar100MWRLS}
\end{table}


\begin{table}[H]
\centering
\scalebox{0.9}{
\begin{tabular}{llrrrr}
\toprule
Branch & Contingency & MVA Flow & AMP Flow & Rate & Loading \\
\midrule
  3001     MINE        230.00   3003     S. MINE     230.00 1 & BUS 151 & 1009.59 & 1012.10 & 300.00 & 337.37 \\
  3001     MINE        230.00   3011     MINE\_G      13.800 1 & BUS 151 & -1576.73 & 1576.73 & 1560.00 & 101.07 \\
  3003     S. MINE     230.00   3005     WEST        230.00 1 & BUS 151 & -482.23 & 506.06 & 350.00 & 144.59 \\
  3003     S. MINE     230.00   3005     WEST        230.00 2 & BUS 151 & -482.23 & 506.06 & 350.00 & 144.59 \\
  3005     WEST        230.00   3007     RURAL       230.00 1 & BUS 151 & -345.24 & 368.28 & 350.00 & 105.22 \\
   154     DOWNTN      230.00    203     EAST230     230.00 1 & BUS 205 & 272.88 & 311.80 & 250.00 & 124.72 \\
\bottomrule
\end{tabular}}
\caption{Bus faults reporting branch flow greater than 100\% for scenario Solar = 100 MW, RLS}
\label{busSolar100MWRLS}
\end{table}


\begin{table}[H]
\centering
\scalebox{0.9}{
\begin{tabular}{llrrrr}
\toprule
Branch & Contingency & MVA Flow & AMP Flow & Rate & Loading \\
\midrule
  3001     MINE        230.00   3003     S. MINE     230.00 1 & SING OPN LIN   1 101-151(1) & 499.44 & 482.94 & 300.00 & 160.98 \\
\bottomrule
\end{tabular}}
\caption{Single line open contingencies reporting branch flow greater than 100\% for scenario Solar = 100 MW, RLS}
\label{singleSolar100MWRLS}
\end{table}

\section{Solar = 100 MW, HLS}


For the studied scenario Solar = 100 MW, HLS, loading greater than 130\% were reported for the branch 3001-3003(1) for the unit fault contingency UNIT 101(1), for the branch 154-203(1) for the bus fault contingency BUS 205, and for the branch 3001-3003(1) for single line open contingencies SING OPN LIN   1 101-151(1).

\begin{table}[H]
\centering
\scalebox{0.88}{
\begin{tabular}{llrrrr}
\toprule
Branch & Contingency & MVA Flow & AMP Flow & Rate & Loading \\
\midrule
  3001     MINE        230.00   3003     S. MINE     230.00 1 & UNIT 101(1) & 543.04 & 526.14 & 300.00 & 175.38 \\
   205     SUB230      230.00    206     URBGEN      18.000 1 & UNIT 211(1) & -1320.38 & 1320.38 & 1250.00 & 105.63 \\
\bottomrule
\end{tabular}}
\caption{Unit faults reporting branch flow greater than 100\% for scenario Solar = 100 MW, HLS}
\label{unitSolar100MWHLS}
\end{table}
\vspace*{-1em}
\begin{table}[H]
\centering
\scalebox{0.88}{
\begin{tabular}{llrrrr}
\toprule
Branch & Contingency & MVA Flow & AMP Flow & Rate & Loading \\
\midrule
   153     MID230      230.00    154     DOWNTN      230.00 1 & BUS 202 & -310.34 & 359.43 & 350.00 & 102.69 \\
   153     MID230      230.00    154     DOWNTN      230.00 1 & BUS 203 & -342.40 & 351.91 & 350.00 & 100.55 \\
   153     MID230      230.00    154     DOWNTN      230.00 1 & BUS 205 & -312.22 & 370.53 & 350.00 & 105.87 \\
   154     DOWNTN      230.00    203     EAST230     230.00 1 & BUS 205 & 284.22 & 337.30 & 250.00 & 134.92 \\
\bottomrule
\end{tabular}}
\caption{Bus faults reporting branch flow greater than 100\% for scenario Solar = 100 MW, HLS}
\label{busSolar100MWHLS}
\end{table}
\vspace*{-1em}
\begin{table}[H]
\centering
\scalebox{0.88}{
\begin{tabular}{llrrrr}
\toprule
Branch & Contingency & MVA Flow & AMP Flow & Rate & Loading \\
\midrule
  3001     MINE        230.00   3003     S. MINE     230.00 1 & SING OPN LIN   1 101-151(1) & 543.04 & 526.14 & 300.00 & 175.38 \\
   154     DOWNTN      230.00    205     SUB230      230.00 1 & SING OPN LIN   6 152-153(1) & 663.58 & 685.92 & 660.00 & 103.93 \\
   153     MID230      230.00    154     DOWNTN      230.00 1 & SING OPN LIN   10 201-205(\&1) & -316.02 & 358.47 & 350.00 & 102.42 \\
   205     SUB230      230.00    206     URBGEN      18.000 1 & SING OPN LIN   11 201-211(1) & -1320.36 & 1320.36 & 1250.00 & 105.63 \\
   153     MID230      230.00    154     DOWNTN      230.00 1 & SING OPN LIN   12 202-203(1) & -344.05 & 353.93 & 350.00 & 101.12 \\
\bottomrule
\end{tabular}}
\caption{Single line open contingencies reporting branch flow greater than 100\% for scenario Solar = 100 MW, HLS}
\label{singleSolar100MWHLS}
\end{table}
\vspace*{-1em}
\section{Results Summary}
The Table ~\ref{sum_load} summarises the results of the AC contingency calculation carried out on the hypothetical SAVNW system for Topology 1 for 
branch overload violations, by tabulating the summary of branch load violation grouped by scenario and contingency to give the branches overloaded and count of the branches overloaded. 
It can be seen that the scenario and contingency for which system is most overloaded is the scenario Solar = 50 MW, RLS and bus fault BUS 151, resulting in 7 branch load violations. The system scenario for the bus fault needs to be studied for long term planning for system reinforcements, and operational planning for alternate dispatch arrangements, demand side measures to ensure system operates within the specified tolerance levels such a contingency occurs. The table gives the branch overload violation count in descending order, and as can be seen the scenarios with the greatest number of violations grouped by Scenario and contingency followed by Solar = 50 MW, RLS \& BUS 151 are the Solar = 0 MW, LLS \& BUS 151, Solar = 100 MW, RLS \& BUS 151 with 5 violations and Solar = 50 MW, LLS \& BUS 151 with 4 violations. As can be seen from the Appendix, Table ~\ref{sum_cont_count}, the contingency reporting the most number of load violation is the bus fault BUS 151 with 24, followed by the  bus fault BUS 205. The single open line contingency reporting the most number of overload violation is the SING OPN LIN 1 101-151(1) with 9 violations. The unit fault contingency with most number of overload violation is the UNIT 101(1) with 9 violations. 
The branches reporting the most number of violations are 3001-3003(1) and 153-154(1) which reported a loading violation for 27 times each followed by 154-203(1) with 9, 3005-3007(1) with 8, and 205-206(1) with 8 violations respectively. The Table ~\ref{sum_branch_count} in the Appendix gives count of branches exceeding the overload $ > 100\% $. 

\begin{table}[H]
\centering
\scalebox{0.72}{\begin{tabular}{llp{9cm}c}
\toprule
Scenario & Contingency & Branches & Branch Count \\
\midrule
Solar = 50 MW, RLS & BUS 151 & 153-3006(1),3001-3003(1),3001-3011(1),3003-3005(1),3003-3005(2),3005-3006(1),3005-3007(1) & 7 \\
Solar = 0 MW, LLS & BUS 151 & 3001-3003(1),3001-3011(1),3003-3005(1),3003-3005(2),3005-3007(1) & 5 \\
Solar = 100 MW, RLS & BUS 151 & 3001-3003(1),3001-3011(1),3003-3005(1),3003-3005(2),3005-3007(1) & 5 \\
Solar = 50 MW, LLS & BUS 151 & 3001-3003(1),3003-3005(1),3003-3005(2),3005-3007(1) & 4 \\
Solar = 50 MW, LLS & SING OPN LIN   15 205-206(1) & 153-154(1),3001-3003(1),3005-3007(1) & 3 \\
Solar = 100 MW, LLS & BUS 151 & 3001-3003(1),3003-3005(1),3003-3005(2) & 3 \\
Solar = 100 MW, LLS & SING OPN LIN   15 205-206(1) & 153-154(1),3001-3003(1),3005-3007(1) & 3 \\
Solar = 100 MW, LLS & UNIT 206(1) & 153-154(1),3001-3003(1),3005-3007(1) & 3 \\
Solar = 50 MW, LLS & UNIT 206(1) & 153-154(1),3001-3003(1),3005-3007(1) & 3 \\
Solar = 0 MW, RLS & BUS 205 & 153-154(1),154-203(1) & 2 \\
Solar = 100 MW, HLS & BUS 205 & 153-154(1),154-203(1) & 2 \\
Solar = 50 MW, HLS & BUS 205 & 153-154(1),154-203(1) & 2 \\
Solar = 50 MW, RLS & BUS 205 & 153-154(1),154-203(1) & 2 \\
Solar = 0 MW, HLS & BUS 205 & 153-154(1),154-203(1) & 2 \\
Solar = 50 MW, HLS & SING OPN LIN   1 101-151(1) & 3001-3003(1) & 1 \\
Solar = 100 MW, RLS & BUS 205 & 154-203(1) & 1 \\
Solar = 100 MW, LLS & UNIT 101(1) & 3001-3003(1) & 1 \\
Solar = 100 MW, RLS & SING OPN LIN   1 101-151(1) & 3001-3003(1) & 1 \\
Solar = 100 MW, RLS & UNIT 101(1) & 3001-3003(1) & 1 \\
Solar = 100 MW, LLS & SING OPN LIN   1 101-151(1) & 3001-3003(1) & 1 \\
Solar = 50 MW, HLS & BUS 202 & 153-154(1) & 1 \\
Solar = 50 MW, HLS & BUS 203 & 153-154(1) & 1 \\
Solar = 0 MW, HLS & BUS 153 & 154-205(1) & 1 \\
Solar = 50 MW, HLS & SING OPN LIN   6 152-153(1) & 154-205(1) & 1 \\
Solar = 50 MW, HLS & SING OPN LIN   10 201-205(\&1) & 153-154(1) & 1 \\
Solar = 50 MW, HLS & SING OPN LIN   11 201-211(1) & 205-206(1) & 1 \\
Solar = 50 MW, HLS & SING OPN LIN   12 202-203(1) & 153-154(1) & 1 \\
Solar = 50 MW, HLS & UNIT 101(1) & 3001-3003(1) & 1 \\
Solar = 50 MW, HLS & UNIT 211(1) & 205-206(1) & 1 \\
Solar = 50 MW, LLS & BUS 205 & 154-203(1) & 1 \\
Solar = 50 MW, LLS & SING OPN LIN   1 101-151(1) & 3001-3003(1) & 1 \\
Solar = 50 MW, LLS & UNIT 101(1) & 3001-3003(1) & 1 \\
Solar = 50 MW, RLS & BUS 203 & 153-154(1) & 1 \\
Solar = 50 MW, RLS & SING OPN LIN   1 101-151(1) & 3001-3003(1) & 1 \\
Solar = 50 MW, RLS & SING OPN LIN   12 202-203(1) & 153-154(1) & 1 \\
Solar = 50 MW, RLS & SING OPN LIN   6 152-153(1) & 154-205(1) & 1 \\
Solar = 100 MW, LLS & BUS 205 & 154-203(1) & 1 \\
Solar = 100 MW, HLS & UNIT 101(1) & 3001-3003(1) & 1 \\
Solar = 100 MW, HLS & UNIT 211(1) & 205-206(1) & 1 \\
Solar = 0 MW, HLS & BUS 203 & 153-154(1) & 1 \\
Solar = 0 MW, HLS & SING OPN LIN   1 101-151(1) & 3001-3003(1) & 1 \\
Solar = 0 MW, HLS & SING OPN LIN   10 201-205(\&1) & 153-154(1) & 1 \\
Solar = 0 MW, HLS & SING OPN LIN   11 201-211(1) & 205-206(1) & 1 \\
Solar = 0 MW, HLS & SING OPN LIN   12 202-203(1) & 153-154(1) & 1 \\
Solar = 0 MW, HLS & SING OPN LIN   2 102-151(1) & 101-151(1) & 1 \\
Solar = 0 MW, HLS & SING OPN LIN   6 152-153(1) & 154-205(1) & 1 \\
Solar = 0 MW, HLS & UNIT 101(1) & 3001-3003(1) & 1 \\
Solar = 0 MW, HLS & UNIT 102(1) & 101-151(1) & 1 \\
Solar = 0 MW, HLS & UNIT 211(1) & 205-206(1) & 1 \\
Solar = 0 MW, LLS & BUS 203 & 153-154(1) & 1 \\
Solar = 0 MW, LLS & BUS 205 & 154-203(1) & 1 \\
Solar = 0 MW, LLS & SING OPN LIN   1 101-151(1) & 3001-3003(1) & 1 \\
Solar = 0 MW, LLS & SING OPN LIN   12 202-203(1) & 153-154(1) & 1 \\
Solar = 0 MW, LLS & UNIT 101(1) & 3001-3003(1) & 1 \\
Solar = 0 MW, RLS & BUS 203 & 153-154(1) & 1 \\
Solar = 0 MW, RLS & SING OPN LIN   1 101-151(1) & 3001-3003(1) & 1 \\
Solar = 0 MW, RLS & SING OPN LIN   10 201-205(\&1) & 153-154(1) & 1 \\
Solar = 0 MW, RLS & SING OPN LIN   11 201-211(1) & 205-206(1) & 1 \\
Solar = 0 MW, RLS & SING OPN LIN   12 202-203(1) & 153-154(1) & 1 \\
Solar = 0 MW, RLS & SING OPN LIN   6 152-153(1) & 154-205(1) & 1 \\
Solar = 0 MW, RLS & UNIT 101(1) & 3001-3003(1) & 1 \\
Solar = 0 MW, RLS & UNIT 211(1) & 205-206(1) & 1 \\
Solar = 100 MW, HLS & BUS 202 & 153-154(1) & 1 \\
Solar = 100 MW, HLS & BUS 203 & 153-154(1) & 1 \\
Solar = 100 MW, HLS & SING OPN LIN   1 101-151(1) & 3001-3003(1) & 1 \\
Solar = 100 MW, HLS & SING OPN LIN   10 201-205(\&1) & 153-154(1) & 1 \\
Solar = 100 MW, HLS & SING OPN LIN   11 201-211(1) & 205-206(1) & 1 \\
Solar = 100 MW, HLS & SING OPN LIN   12 202-203(1) & 153-154(1) & 1 \\
Solar = 100 MW, HLS & SING OPN LIN   6 152-153(1) & 154-205(1) & 1 \\
Solar = 50 MW, RLS & UNIT 101(1) & 3001-3003(1) & 1 \\
\bottomrule
\end{tabular}}
\caption{Summary of branch load violation grouped by Scenario and Contingency}
\label{sum_load}
\end{table}

\chapter{Upper Emergency Bus Voltage Violation}
\label{upper}
\section{Introduction}
In this chapter the upper voltage limit violations are reported by bus faults, unit faults and single line faults for each of the studied scenario.
The violations are reported in tabular format and in the reported table, Base Voltage is PU base case voltage, Contingency Voltage is PU contingency case voltage, Deviation is difference between contingency case and base case voltage, Range Violation is range violations calculated as Contingency Voltage - maximum range limit (1.1 PU for the upper emergency range). It was seen that for Topology 1, only the scenarios, Solar = 50 MW, LLS, Solar = 100 MW, LLS reported upper voltage limit violations.

\section{Solar = 50 MW, LLS}


For the studied scenario, Solar = 50 MW, LLS, the bus reporting the upper emergency range violation is the bus 211 for the unit fault UNIT 206(1) and the single line open fault SING OPN LIN   15 205-206(1). 

\begin{table}[H]
\centering
\scalebox{0.9}{
\begin{tabular}{llrrrr}
\toprule
Bus Number & Contingency & Base Voltage & Contingency Voltage & Deviation & Range Violation \\
\midrule
211 & UNIT 206(1) & 1.04 & 1.11 & 0.07 & 0.06 \\
\bottomrule
\end{tabular}}
\caption{Unit faults reporting bus voltages greater than emergency voltage of 1.1 PU for scenario Solar = 50 MW, LLS}
\label{hvunitSolar50MWLLS}
\end{table}


\begin{table}[H]
\centering
\scalebox{0.9}{
\begin{tabular}{llrrrr}
\toprule
Bus Number & Contingency & Base Voltage & Contingency Voltage & Deviation & Range Violation \\
\midrule
211 & SING OPN LIN   15 205-206(1) & 1.04 & 1.11 & 0.07 & 0.06 \\
\bottomrule
\end{tabular}}
\caption{Single line open contingencies reporting bus voltages greater than emergency voltage of 1.1 PU for scenario Solar = 50 MW, LLS}
\label{hvsingleSolar50MWLLS}
\end{table}

\section{Solar = 100 MW, LLS}

For the studied scenario Solar = 100 MW, LLS, the bus violating the upper voltage limit is the bus 211 for the unit fault UNIT 206(1) and the single line open fault SING OPN LIN   15 205-206(1).

\begin{table}[H]
\centering
\scalebox{0.9}{
\begin{tabular}{llrrrr}
\toprule
Bus Number & Contingency & Base Voltage & Contingency Voltage & Deviation & Range Violation \\
\midrule
211 & UNIT 206(1) & 1.04 & 1.11 & 0.07 & 0.06 \\
\bottomrule
\end{tabular}}
\caption{Unit faults reporting bus voltages greater than emergency voltage of 1.1 PU for scenario Solar = 100 MW, LLS}
\label{hvunitSolar100MWLLS}
\end{table}


\begin{table}[H]
\centering
\scalebox{0.9}{
\begin{tabular}{llrrrr}
\toprule
Bus Number & Contingency & Base Voltage & Contingency Voltage & Deviation & Range Violation \\
\midrule
211 & SING OPN LIN   15 205-206(1) & 1.04 & 1.11 & 0.07 & 0.06 \\
\bottomrule
\end{tabular}}
\caption{Single line open contingencies reporting bus voltages greater than emergency voltage of 1.1 PU for scenario Solar = 100 MW, LLS}
\label{hvsingleSolar100MWLLS}
\end{table}

\section{Results Summary}

It can be seen that there is only the Bus 211 reporting the upper voltage range violation. The bus 211 reports upper voltage range violation for the single open line contingency of SING OPN LIN   15 205-206(1) and for the unit fault UNIT 206(1) for two scenarios Solar = 100 MW, LLS and Solar = 50 MW, LLS. 

\begin{table}[H]
\centering
\scalebox{0.9}{

\begin{tabular}{lllr}
\toprule
Scenario & Contingency & Buses & Bus Count \\
\midrule
Solar = 100 MW, LLS & SING OPN LIN   15 205-206(1) & 211 & 1 \\
Solar = 100 MW, LLS & UNIT 206(1) & 211 & 1 \\
Solar = 50 MW, LLS & SING OPN LIN   15 205-206(1) & 211 & 1 \\
Solar = 50 MW, LLS & UNIT 206(1) & 211 & 1 \\
\bottomrule
\end{tabular}}
\caption{Summary of upper limit voltage violation grouped by Scenario and Contingency}
\label{sum_high}
\end{table}


\chapter{Lower Emergency Bus Voltage Violation}
\label{lower}
\section{Introduction}
In this chapter the lower voltage limit violations are reported by bus faults, unit faults and single line faults for each of the studied scenario.
The violations are reported in tabular format and in the reported table, Base Voltage is PU base case voltage, Contingency Voltage is PU contingency case voltage, Deviation is difference between contingency case and base case voltage, Range Violation is range violations calculated as Contingency Voltage - minimum range limit (0.9 PU for lower emergency range).

\section{Solar = 0 MW, LLS}
For the studied scenario Solar = 0 MW, LLS, the buses violating the lower voltage limits are buses 103, 154 for the bus fault BUS 205. 

\begin{table}[H]
\centering
\scalebox{0.9}{
\begin{tabular}{llrrrr}
\toprule
Bus Number & Contingency & Base Voltage & Contingency Voltage & Deviation & Range Violation \\
\midrule
103 & BUS 205 & 0.977 & 0.895 & -0.082 & -0.055 \\
154 & BUS 205 & 0.977 & 0.895 & -0.082 & -0.055 \\
\bottomrule
\end{tabular}}
\caption{Bus faults reporting bus voltages lower than emergency voltage of 0.9 PU for scenario Solar = 0 MW, LLS}
\label{lvbusSolar0MWLLS}
\end{table}

\section{Solar = 0 MW, RLS}
For the studied scenario Solar = 0 MW, RLS, the buses violating the lower voltage limits are Buses 203, 103, 154, 3008.
Bus 203 reported lower emergency range violation for the bus fault BUS 202 and the buses 103, 154, 3008 reported lower emergency range violations for BUS 205. 

\begin{table}[H]
\centering
\scalebox{0.9}{
\begin{tabular}{llrrrr}
\toprule
Bus Number & Contingency & Base Voltage & Contingency Voltage & Deviation & Range Violation \\
\midrule
203 & BUS 202 & 1.000 & 0.893 & -0.107 & -0.057 \\
103 & BUS 205 & 0.975 & 0.857 & -0.118 & -0.093 \\
154 & BUS 205 & 0.975 & 0.857 & -0.118 & -0.093 \\
3008 & BUS 205 & 0.993 & 0.897 & -0.096 & -0.053 \\
\bottomrule
\end{tabular}}
\caption{Bus faults reporting bus voltages lower than emergency voltage of 0.9 PU for scenario Solar = 0 MW, RLS}
\label{lvbusSolar0MWRLS}
\end{table}

\section{Solar = 0 MW, HLS}
For the studied scenario Solar = 0 MW, HLS, the buses violating the lower voltage limits are Buses 103, 154, 203, 205, 3008, 3007
For the bus fault BUS 205, the buses reporting lower emergency range violations are 103, 154, 203, 3007, 3008
\begin{table}[H]
\centering
\scalebox{0.9}{
\begin{tabular}{llrrrr}
\toprule
Bus Number & Contingency & Base Voltage & Contingency Voltage & Deviation & Range Violation \\
\midrule
103 & BUS 205 & 0.973 & 0.819 & -0.154 & -0.131 \\
154 & BUS 205 & 0.973 & 0.819 & -0.154 & -0.131 \\
203 & BUS 205 & 0.998 & 0.898 & -0.100 & -0.052 \\
3007 & BUS 205 & 0.993 & 0.885 & -0.108 & -0.065 \\
3008 & BUS 205 & 0.990 & 0.865 & -0.125 & -0.085 \\
\bottomrule
\end{tabular}}
\caption{Bus faults reporting bus voltages lower than emergency voltage of 0.9 PU for scenario Solar = 0 MW, HLS}
\label{lvbusSolar0MWHLS}
\end{table}

For the single line fault SING OPN LIN   10 201-205(\&1), the buses reporting lower emergency range violations are 103, 154, 203, 205, 3008
\begin{table}[H]
\centering
\scalebox{0.9}{
\begin{tabular}{llrrrr}
\toprule
Bus Number & Contingency & Base Voltage & Contingency Voltage & Deviation & Range Violation \\
\midrule
103 & SING OPN LIN   10 201-205(\&1) & 0.973 & 0.847 & -0.126 & -0.103 \\
154 & SING OPN LIN   10 201-205(\&1) & 0.973 & 0.847 & -0.126 & -0.103 \\
203 & SING OPN LIN   10 201-205(\&1) & 0.998 & 0.887 & -0.111 & -0.063 \\
205 & SING OPN LIN   10 201-205(\&1) & 0.980 & 0.852 & -0.128 & -0.098 \\
3008 & SING OPN LIN   10 201-205(\&1) & 0.990 & 0.885 & -0.105 & -0.065 \\
\bottomrule
\end{tabular}}
\caption{Single line open contingencies reporting bus voltages lower than emergency voltage of 0.9 PU for scenario Solar = 0 MW, HLS}
\label{lvsingleSolar0MWHLS}
\end{table}

\section{Solar = 50 MW, LLS}
For the studied scenario Solar = 50 MW, LLS, the buses violating the lower voltage limits are Buses 103, 154, 205, 3008, 206
For the unit fault UNIT 206(1), the buses reporting lower emergency range violations are 103, 154, 205, 206, 3008
\begin{table}[H]
\centering
\scalebox{0.9}{
\begin{tabular}{llrrrr}
\toprule
Bus Number & Contingency & Base Voltage & Contingency Voltage & Deviation & Range Violation \\
\midrule
103 & UNIT 206(1) & 0.976 & 0.861 & -0.115 & -0.089 \\
154 & UNIT 206(1) & 0.977 & 0.862 & -0.115 & -0.088 \\
205 & UNIT 206(1) & 0.980 & 0.864 & -0.116 & -0.086 \\
206 & UNIT 206(1) & 0.996 & 0.864 & -0.132 & -0.086 \\
3008 & UNIT 206(1) & 0.997 & 0.893 & -0.104 & -0.057 \\
\bottomrule
\end{tabular}}
\caption{Unit faults reporting bus voltages lower than emergency voltage of 0.9 PU for scenario Solar = 50 MW, LLS}
\label{lvunitSolar50MWLLS}
\end{table}

For the single line fault SING OPN LIN   15 205-206(1), the buses reporting lower emergency range violations are 103, 154, 205, 3008
\begin{table}[H]
\centering
\scalebox{0.85}{
\begin{tabular}{llrrrr}
\toprule
Bus Number & Contingency & Base Voltage & Contingency Voltage & Deviation & Range Violation \\
\midrule
103 & SING OPN LIN   15 205-206(1) & 0.976 & 0.861 & -0.115 & -0.089 \\
154 & SING OPN LIN   15 205-206(1) & 0.977 & 0.862 & -0.115 & -0.088 \\
205 & SING OPN LIN   15 205-206(1) & 0.980 & 0.864 & -0.116 & -0.086 \\
3008 & SING OPN LIN   15 205-206(1) & 0.997 & 0.893 & -0.104 & -0.057 \\
\bottomrule
\end{tabular}}
\caption{Single line open contingencies reporting bus voltages lower than emergency voltage of 0.9 PU for scenario Solar = 50 MW, LLS}
\label{lvsingleSolar50MWLLS}
\end{table}

\section{Solar = 50 MW, RLS}
For the studied scenario Solar = 50 MW, RLS, the buses violating the lower voltage limits are Buses 103, 153, 154, 203, 205, 3004, 3005, 3006, 3007, 3008
For the bus fault BUS 151, the buses reporting lower emergency range violations are 103, 153, 154, 203, 205, 3004, 3005, 3006, 3007, 3008
\begin{table}[H]
\centering
\scalebox{0.85}{
\begin{tabular}{llrrrr}
\toprule
Bus Number & Contingency & Base Voltage & Contingency Voltage & Deviation & Range Violation \\
\midrule
103 & BUS 151 & 0.974 & 0.856 & -0.118 & -0.094 \\
153 & BUS 151 & 1.024 & 0.881 & -0.143 & -0.069 \\
154 & BUS 151 & 0.975 & 0.858 & -0.117 & -0.092 \\
203 & BUS 151 & 1.001 & 0.891 & -0.110 & -0.059 \\
205 & BUS 151 & 0.980 & 0.877 & -0.103 & -0.073 \\
3004 & BUS 151 & 1.043 & 0.882 & -0.161 & -0.068 \\
3005 & BUS 151 & 1.022 & 0.860 & -0.162 & -0.090 \\
3006 & BUS 151 & 1.024 & 0.871 & -0.153 & -0.079 \\
3007 & BUS 151 & 0.997 & 0.836 & -0.161 & -0.114 \\
3008 & BUS 151 & 0.994 & 0.845 & -0.149 & -0.105 \\
\bottomrule
\end{tabular}}
\caption{Bus faults reporting bus voltages lower than emergency voltage of 0.9 PU for scenario Solar = 50 MW, RLS}
\label{lvbusSolar50MWRLS}
\end{table}

\section{Solar = 50 MW, HLS}
For the studied scenario Solar = 50 MW, HLS, the buses violating the lower voltage limits are Buses 103, 154, 205, 203, 3008, 3007
For the bus fault BUS 202, the buses reporting lower emergency range violations are 103, 154, 203, 205, 3008
For the bus fault BUS 205, the buses reporting lower emergency range violations are 3007
\begin{table}[H]
\centering
\scalebox{0.85}{
\begin{tabular}{llrrrr}
\toprule
Bus Number & Contingency & Base Voltage & Contingency Voltage & Deviation & Range Violation \\
\midrule
103 & BUS 202 & 0.973 & 0.849 & -0.124 & -0.101 \\
154 & BUS 202 & 0.974 & 0.850 & -0.123 & -0.100 \\
203 & BUS 202 & 0.999 & 0.833 & -0.166 & -0.117 \\
205 & BUS 202 & 0.980 & 0.858 & -0.122 & -0.092 \\
3008 & BUS 202 & 0.991 & 0.890 & -0.100 & -0.060 \\
3007 & BUS 205 & 0.994 & 0.896 & -0.098 & -0.054 \\
\bottomrule
\end{tabular}}
\caption{Bus faults reporting bus voltages lower than emergency voltage of 0.9 PU for scenario Solar = 50 MW, HLS}
\label{lvbusSolar50MWHLS}
\end{table}

For the single line fault SING OPN LIN   10 201-205(\&1), the buses reporting lower emergency range violations are 103, 154, 205
\begin{table}[H]
\centering
\scalebox{0.85}{
\begin{tabular}{llrrrr}
\toprule
Bus Number & Contingency & Base Voltage & Contingency Voltage & Deviation & Range Violation \\
\midrule
103 & SING OPN LIN   10 201-205(\&1) & 0.973 & 0.867 & -0.105 & -0.083 \\
154 & SING OPN LIN   10 201-205(\&1) & 0.974 & 0.868 & -0.105 & -0.082 \\
205 & SING OPN LIN   10 201-205(\&1) & 0.980 & 0.874 & -0.106 & -0.076 \\
\bottomrule
\end{tabular}}
\caption{Single line open contingencies reporting bus voltages lower than emergency voltage of 0.9 PU for scenario Solar = 50 MW, HLS}
\label{lvsingleSolar50MWHLS}
\end{table}

\section{Solar = 100 MW, LLS}
For the studied scenario Solar = 100 MW, LLS, the buses violating the lower voltage limits are Buses 103, 154, 205, 3008, 206
For the unit fault UNIT 206(1), the buses reporting lower emergency range violations are 103, 154, 205, 206, 3008
\begin{table}[H]
\centering
\scalebox{0.85}{
\begin{tabular}{llrrrr}
\toprule
Bus Number & Contingency & Base Voltage & Contingency Voltage & Deviation & Range Violation \\
\midrule
103 & UNIT 206(1) & 0.974 & 0.861 & -0.112 & -0.089 \\
154 & UNIT 206(1) & 0.977 & 0.867 & -0.111 & -0.083 \\
205 & UNIT 206(1) & 0.980 & 0.869 & -0.111 & -0.081 \\
206 & UNIT 206(1) & 0.994 & 0.869 & -0.126 & -0.081 \\
3008 & UNIT 206(1) & 0.997 & 0.897 & -0.101 & -0.053 \\
\bottomrule
\end{tabular}}
\caption{Unit faults reporting bus voltages lower than emergency voltage of 0.9 PU for scenario Solar = 100 MW, LLS}
\label{lvunitSolar100MWLLS}
\end{table}

For the single line fault SING OPN LIN   15 205-206(1), the buses reporting lower emergency range violations are 103, 154, 205, 3008
\begin{table}[H]
\centering
\scalebox{0.9}{
\begin{tabular}{llrrrr}
\toprule
Bus Number & Contingency & Base Voltage & Contingency Voltage & Deviation & Range Violation \\
\midrule
103 & SING OPN LIN   15 205-206(1) & 0.974 & 0.861 & -0.112 & -0.089 \\
154 & SING OPN LIN   15 205-206(1) & 0.977 & 0.867 & -0.111 & -0.083 \\
205 & SING OPN LIN   15 205-206(1) & 0.980 & 0.869 & -0.111 & -0.081 \\
3008 & SING OPN LIN   15 205-206(1) & 0.997 & 0.897 & -0.101 & -0.053 \\
\bottomrule
\end{tabular}}
\caption{Single line open contingencies reporting bus voltages lower than emergency voltage of 0.9 PU for scenario Solar = 100 MW, LLS}
\label{lvsingleSolar100MWLLS}
\end{table}

\section{Solar = 100 MW, RLS}
For the studied scenario Solar = 100 MW, RLS, the buses violating the lower voltage limits are Buses 103, 154
For the bus fault BUS 205, the buses reporting lower emergency range violations are 103, 154
\begin{table}[H]
\centering
\scalebox{0.9}{
\begin{tabular}{llrrrr}
\toprule
Bus Number & Contingency & Base Voltage & Contingency Voltage & Deviation & Range Violation \\
\midrule
103 & BUS 205 & 0.972 & 0.870 & -0.102 & -0.080 \\
154 & BUS 205 & 0.975 & 0.875 & -0.100 & -0.075 \\
\bottomrule
\end{tabular}}
\caption{Bus faults reporting bus voltages lower than emergency voltage of 0.9 PU for scenario Solar = 100 MW, RLS}
\label{lvbusSolar100MWRLS}
\end{table}

\section{Solar = 100 MW, HLS}
For the studied scenario Solar = 100 MW, HLS, the buses violating the lower voltage limits are Buses 103, 154, 205, 203, 3008
For the bus fault BUS 202, the buses reporting lower emergency range violations are 103, 154, 203, 205
For the bus fault BUS 205, the buses reporting lower emergency range violations are 3008
\begin{table}[H]
\centering
\scalebox{0.9}{
\begin{tabular}{llrrrr}
\toprule
Bus Number & Contingency & Base Voltage & Contingency Voltage & Deviation & Range Violation \\
\midrule
103 & BUS 202 & 0.970 & 0.858 & -0.112 & -0.092 \\
154 & BUS 202 & 0.974 & 0.863 & -0.110 & -0.087 \\
203 & BUS 202 & 1.000 & 0.847 & -0.153 & -0.103 \\
205 & BUS 202 & 0.980 & 0.871 & -0.109 & -0.079 \\
3008 & BUS 205 & 0.991 & 0.885 & -0.106 & -0.065 \\
\bottomrule
\end{tabular}}
\caption{Bus faults reporting bus voltages lower than emergency voltage of 0.9 PU for scenario Solar = 100 MW, HLS}
\label{lvbusSolar100MWHLS}
\end{table}

For the single line fault SING OPN LIN   10 201-205(\&1), the buses reporting lower emergency range violations are 103, 154, 205
\begin{table}[H]
\centering
\scalebox{0.9}{
\begin{tabular}{llrrrr}
\toprule
Bus Number & Contingency & Base Voltage & Contingency Voltage & Deviation & Range Violation \\
\midrule
103 & SING OPN LIN   10 201-205(\&1) & 0.970 & 0.876 & -0.094 & -0.074 \\
154 & SING OPN LIN   10 201-205(\&1) & 0.974 & 0.882 & -0.092 & -0.068 \\
205 & SING OPN LIN   10 201-205(\&1) & 0.980 & 0.887 & -0.093 & -0.063 \\
\bottomrule
\end{tabular}}
\caption{Single line open contingencies reporting bus voltages lower than emergency voltage of 0.9 PU for scenario Solar = 100 MW, HLS}
\label{lvsingleSolar100MWHLS}
\end{table}


\section{Results Summary}

The Table ~\ref{sum_low} summarises the results of the AC contingency calculation carried out on the hypothetical SAVNW system for Topology 1 for lower emergency bus voltage violations by tabulating the summary of lower voltage violations grouped by scenario and contingency to give the buses reporting lower voltage range violation and count of the lower voltage range violations. 

It can be seen that the scenario Solar = 50 MW, RLS and bus fault BUS 151 reports the most number of buses violating the lower emergency limit of 0.9 PU with 10 buses. For scenarios and faults Solar = 0 MW, HLS \& BUS 205, Solar = 0 MW, HLS \& SING OPN LIN 10 201-205(\&1), Solar = 50 MW, LLS \& UNIT 206(1), Solar = 50 MW, HLS \& BUS 202, Solar = 100 MW, LLS \& UNIT 206(1) there were 5 lower limit violations reported. 

The buses reporting most number of lower voltage limit violations are the buses 103 \& 154 with 34 violations, followed by buses 3008, 205 with 22 and 20 violations respectively. Further transmission planning studies needed to be carried out for system reinforcements to ensure reliable and stable operation of the system should a contingency causes system to violate the lower voltage limit. The Table ~\ref{sum_low_count} in the Appendix gives the count of times the buses violate the lower voltage limit $ < 0.9 PU $. 

The contingency causing the most number of lower voltage limit violation is the bus fault BUS 205 resulting in 42 lower limit violation across all the studied scenarios followed by SING OPN LIN 10 201-205(\&1) with 22 violations, BUS 202, UNIT 206(1), and BUS 151 with 20 limit violations. The Table ~\ref{sum_low_cont_count} in the Appendix gives the contingencies and no of times these contingencies cause a lower voltage limit violation $ < 0.9 PU $. 



\begin{table}[H]
\centering
\scalebox{0.9}{\begin{tabular}{lllr}
\toprule
Scenario & Contingency & Buses & Bus Count \\
\midrule
Solar = 50 MW, RLS & BUS 151 & 103,153,154,203,205,3004,3005,3006,3007,3008 & 10 \\
Solar = 0 MW, HLS & BUS 205 & 103,154,203,3007,3008 & 5 \\
Solar = 0 MW, HLS & SING OPN LIN   10 201-205(\&1) & 103,154,203,205,3008 & 5 \\
Solar = 50 MW, LLS & UNIT 206(1) & 103,154,205,206,3008 & 5 \\
Solar = 50 MW, HLS & BUS 202 & 103,154,203,205,3008 & 5 \\
Solar = 100 MW, LLS & UNIT 206(1) & 103,154,205,206,3008 & 5 \\
Solar = 100 MW, LLS & SING OPN LIN   15 205-206(1) & 103,154,205,3008 & 4 \\
Solar = 100 MW, HLS & BUS 202 & 103,154,203,205 & 4 \\
Solar = 50 MW, HLS & BUS 205 & 103,154,3007,3008 & 4 \\
Solar = 50 MW, LLS & SING OPN LIN   15 205-206(1) & 103,154,205,3008 & 4 \\
Solar = 100 MW, HLS & BUS 205 & 103,154,3008 & 3 \\
Solar = 100 MW, HLS & SING OPN LIN   10 201-205(\&1) & 103,154,205 & 3 \\
Solar = 0 MW, RLS & BUS 205 & 103,154,3008 & 3 \\
Solar = 50 MW, HLS & SING OPN LIN   10 201-205(\&1) & 103,154,205 & 3 \\
Solar = 100 MW, RLS & BUS 205 & 103,154 & 2 \\
Solar = 0 MW, LLS & BUS 205 & 103,154 & 2 \\
Solar = 50 MW, RLS & BUS 205 & 103,154 & 2 \\
Solar = 0 MW, RLS & BUS 202 & 203 & 1 \\
\bottomrule
\end{tabular}}
\caption{Summary of lower limit voltage violation grouped by Scenario and Contingency}
\label{sum_low}
\end{table}


\chapter{Conculsion}
\label{conclusion} 


The contingency analysis was carried out on the hypothetical SAVNW study system for the Year 1, Topology 1 for 9 study scenarios for (N-1) bus, single line open, unit contingencies. Out of all the studied (N-1) contingencies, the contingencies for which the system did not converge are tabulated in Table \ref{count_blown} with the corresponding number of scenarios. 

\begin{table}[H]
\centering
\scalebox{0.9}{
\begin{tabular}{lr}
\toprule
Contingency & Count of Scenarios \\
\midrule
BUS 152 & 9 \\
BUS 154 & 9 \\
BUS 201 & 9 \\
UNIT 206(1) & 7 \\
SING OPN LIN   15 205-206(1) & 7 \\
BUS 151 & 4 \\
BUS 202 & 1 \\
\bottomrule
\end{tabular}}
\caption{Count of scenarios for which the tested contingency failed to converge}
\label{count_blown}
\end{table}


It can be seen from the Table \ref{count_blown}, for the (N-1) contingencies BUS 152, BUS 154, BUS 201 did not converge for any of the 9 studied scenarios. The (N-1) contingencies UNIT 206(1),
SING OPN LIN   15 205-206(1) did not converge for 7 of the studied scenarios, followed by BUS 151 for 4 and BUS 202 for 1 of the studied scenarios. 

For the converged contingency scenarios, Chapter \ref{overload} studied the branch overload violations, Chapter \ref{upper} studied the upper limit voltage violation, and Chapter \ref{lower} studied the lower limit voltage violations. 

In the Chapter \ref{overload} branches reporting loading greater than 100\% were tabulated for each scenario and from the table, branches reporting loading greater than 130\% were also noted down. It was seen that the branch with most number of overload violation were reported for branches 3001-3003(1) and 153-154(1) with 27 violations each.  

In the Chapter \ref{upper} buses reporting voltage greater than 1.1 PU (violation of upper limit) were studied and it was seen that only the generator bus 211reported the upper limit violation. In the Chapter \ref{lower} buses reporting voltage less than 0.9 PU (violation of lower limit) were tabulated, there were multiple buses reporting lower limit violation and the buses 103 and 154 in area 1 reported the most number of violations. 






\chapter*{Appendix}
\addcontentsline{toc}{chapter}{Appendix}
\label{chapter:app}

\section{Branch Overload Violation $ > 100\% $ Counts}

\begin{table}[H]


\centering
\scalebox{0.8}{\begin{tabular}{lr}
\toprule
Contingency Overloaded & Overload Count \\
\midrule
BUS 151 & 24 \\
BUS 205 & 14 \\
SING OPN LIN   1 101-151(1) & 9 \\
UNIT 101(1) & 9 \\
SING OPN LIN   12 202-203(1) & 6 \\
BUS 203 & 6 \\
SING OPN LIN   15 205-206(1) & 6 \\
UNIT 206(1) & 6 \\
SING OPN LIN   6 152-153(1) & 5 \\
SING OPN LIN   10 201-205(\&1) & 4 \\
SING OPN LIN   11 201-211(1) & 4 \\
UNIT 211(1) & 4 \\
BUS 202 & 2 \\
SING OPN LIN   2 102-151(1) & 1 \\
BUS 153 & 1 \\
UNIT 102(1) & 1 \\
\bottomrule
\end{tabular}}
\caption{Count of reported branch overload violations due to contingency}
\label{sum_cont_count}
\end{table}

\begin{table}[H]
\centering
\scalebox{0.8}{
\begin{tabular}{lr}
\toprule
Branch Overloaded & Overload Count \\
\midrule
3001-3003(1) & 27 \\
153-154(1) & 27 \\
3001-3011(1) & 3 \\
3003-3005(1) & 5 \\
3003-3005(2) & 5 \\
3005-3007(1) & 8 \\
154-203(1) & 9 \\
154-205(1) & 6 \\
205-206(1) & 8 \\
101-151(1) & 2 \\
153-3006(1) & 1 \\
3005-3006(1) & 1 \\
\bottomrule
\end{tabular}}
\caption{Count of branches reporting overload violations }
\label{sum_branch_count}
\end{table}

\begin{table}[H]
\centering
\scalebox{0.8}{

\begin{tabular}{lr}
\toprule
Scenario & Overload Counts \\
\midrule
Solar = 50 MW, RLS & 14 \\
Solar = 0 MW, HLS & 13 \\
Solar = 50 MW, LLS & 13 \\
Solar = 100 MW, LLS & 12 \\
Solar = 50 MW, HLS & 11 \\
Solar = 100 MW, HLS & 11 \\
Solar = 0 MW, LLS & 10 \\
Solar = 0 MW, RLS & 10 \\
Solar = 100 MW, RLS & 8 \\
\bottomrule
\end{tabular}}
\caption{Scenario Overload Summary}
\label{scen_load}
\end{table}

\section{Upper Voltage Range Violation ($ > 1.1 PU $) Counts}


\begin{table}[H]
\centering
\scalebox{0.9}{

\begin{tabular}{lr}
\toprule
Scenario & Overload Counts \\
\midrule
Solar = 50 MW, LLS & 4 \\
Solar = 100 MW, LLS & 4 \\
\bottomrule
\end{tabular}}
\caption{Scenario upper voltage limit violation summary}
\label{scen_load}
\end{table}

\section{Lower Voltage Range Violation ($ < 0.9 PU $) Counts}

\begin{table}[H]

\centering
\scalebox{0.9}{
\begin{tabular}{lr}
\toprule
Bus & Lower violation counts \\
\midrule
103 & 34 \\
154 & 34 \\
3008 & 22 \\
205 & 20 \\
203 & 12 \\
3007 & 6 \\
206 & 4 \\
153 & 2 \\
3004 & 2 \\
3005 & 2 \\
3006 & 2 \\
\bottomrule
\end{tabular}}
\caption{Summary of lower voltage range violation - Bus count}
\label{sum_low_count}
\end{table}


\begin{table}[H]
\centering
\scalebox{0.9}{

\begin{tabular}{lr}
\toprule
Contingency & Lower violation count \\
\midrule
BUS 205 & 42 \\
SING OPN LIN   10 201-205(\&1) & 22 \\
BUS 202 & 20 \\
UNIT 206(1) & 20 \\
BUS 151 & 20 \\
SING OPN LIN   15 205-206(1) & 16 \\
\bottomrule
\end{tabular}}
\caption{Summary of lower voltage range violation - Contingency count}
\label{sum_low_cont_count}
\end{table}


\begin{table}[H]
\centering
\scalebox{0.9}{
\begin{tabular}{lr}
\toprule
Scenario & Overload Counts \\
\midrule
Solar = 50 MW, RLS & 24 \\
Solar = 50 MW, HLS & 24 \\
Solar = 0 MW, HLS & 20 \\
Solar = 100 MW, HLS & 20 \\
Solar = 50 MW, LLS & 18 \\
Solar = 100 MW, LLS & 18 \\
Solar = 0 MW, RLS & 8 \\
Solar = 0 MW, LLS & 4 \\
Solar = 100 MW, RLS & 4 \\
\bottomrule
\end{tabular}}
\caption{Scenario lower voltage limit violation summary}
\label{scen_load}
\end{table}


\end{document}